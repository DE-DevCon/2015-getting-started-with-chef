\documentclass{beamer}
\usepackage[utf8]{inputenc}
\usepackage{listings}
\usepackage{multicol}
\usetheme{Berlin}
\setbeamercolor{palette secondary}{use=seahorse}
\usecolortheme{seahorse}
\title[Getting Started with Chef]{Getting Started with Chef}
\author{\hspace{12pt}Robert “Anthony” Bittle\hspace{12pt}\\\hspace{12pt}robert.bittle@dominionenterprises.com\hspace{12pt}\\\hspace{12pt}github.com/guywithnose\hspace{12pt}}
\date{October 8, 2015}
\setbeamertemplate{itemize items}[circle]
\begin{document}
    \begin{frame}
        \titlepage
        Slides and demos at:\\
        \href{https://github.com/guywithnose/2015-getting-started-with-chef}{https://github.com/guywithnose/2015-getting-started-with-chef}
        \href{https://goo.gl/GTsdg7}{https://goo.gl/GTsdg7}
    \end{frame}

    \section{What is Chef?}
    \subsection{What is Chef?}
    \begin{frame}{IT Automation}
        \begin{itemize}
            \item Server Provisioning
            \item Configuration Management
            \item Code Deployment
        \end{itemize}
    \end{frame}
    \begin{frame}{Environment Management}
        \begin{itemize}
            \item Manage a Scalable Cloud Environment
            \item Manage Server Configuration with Version Control
        \end{itemize}
    \end{frame}
    \section{Components}
    \subsection{Components}
    \begin{frame}{Resources}
        Resources are the smallest action item.  Ideally they should only do one thing.\\
        \medskip
        \texttt{directory '/tmp/folder' do\\~~owner 'root'\\~~group 'root'\\~~mode '0755'\\~~action :create\\end\\}
    \end{frame}
    \begin{frame}{Recipes}
        Recipes use resources to accomplish a common task.\\(Example: Install Nginx)
    \end{frame}
    \begin{frame}{Cookbooks}
        Cookbooks are make up of multiple recipes that will generally be used together.\\
        Cookbooks can include other cookbooks.\\
        \medskip
        We're going to go over setting up a simple cookbook that provisions a LAMP server and deploys a specific revision from version control.
    \end{frame}

    \section{Cookbook Directory Structure}
    \subsection{Cookbook Directory Structure}
    \begin{frame}{Recipes}
        $\{MY\_COOKBOOK\_DIRECTORY\}/recipes$\\
        Every file in this directory can be included as a recipe.\\
        \bigskip
        $\{MY\_COOKBOOK\_DIRECTORY\}/recipes/awesome.rb$\\
        Can be included with:\\
        $include\_recipe~~'my\_cookbook::awesome'$\\
        \bigskip
        $\{MY\_COOKBOOK\_DIRECTORY\}/recipes/default.rb$\\
        Can be included with:\\
        $include\_recipe~~'my\_cookbook'$\\
    \end{frame}
    \begin{frame}{Attributes}
        $\{MY\_COOKBOOK\_DIRECTORY\}/attributes/default.rb$\\
        This is where all your default configuration and settings go.\\
    \end{frame}
    \begin{frame}{Custom Resources and Providers}
        $\{MY\_COOKBOOK\_DIRECTORY\}/resources$\\
        $\{MY\_COOKBOOK\_DIRECTORY\}/providers$\\
        \bigskip
        This is where you can implement your own resources, extending chef to accomplish a specific task.\\
        \bigskip
        In general each resource file has a corresponding provider file.\\
        \bigskip
        Resources are similar to header files in that they describe the interface of a resource and the provider file contains the implementation.\\
    \end{frame}
    \section{Dependency Management}
    \subsection{Dependency Management}
    \begin{frame}{Including Recipes From Other Cookbooks}
        As you develop your own recipes you will want to depend on other recipes.  For example a LAMP could start with this:
        \medskip
        \texttt{
            include\_recipe 'mysql'\\
            include\_recipe 'php'\\
            include\_recipe 'apache2'\\
        }
    \end{frame}
    \begin{frame}{These Other Cookbooks Need Other Cookbooks}
        You could manually download each of these recipes, but this quickly becomes impractical.\\
        \bigskip
        For example the php recipe depends on all of these recipes:\\
        build-essential\\
        iis\\
        mysql\\
        windows\\
        xml\\
        yum-epel\\
    \end{frame}
    \begin{frame}{Berkshelf Saves the Day}
        To make this more practical you'll need to use \textbf{Berkshelf} for cookbook dependency management.\\
        To include recipes you need to create a metadata file very similar to a composer.json, package.js, Gemfile, etc.\\
        \medskip
        \texttt{depends 'mysql', '\textasciitilde> 6.0'\\
            depends 'apache2'\\
            depends 'php'\\
        }
    \end{frame}
    \section{A Simple Example}
    \subsection{A Simple Example}
    \begin{frame}{Set Up MySQL}
        \texttt{inclue\_recipe 'mysql'\\
            mysql\_service 'demo' do\\
            ~~port '3306'\\
            ~~version '5.6'\\
            ~~data\_dir '/data'\\
            ~~initial\_root\_password 'suPer\$3crEt'\\
            ~~action [:create, :start]\\
            end\\
        }
    \end{frame}
    \begin{frame}{Set Up PHP}
        \texttt{include\_recipe 'php'\\
            include\_recipe 'php::module\_mysql'\\
        }
    \end{frame}
    \begin{frame}{Set Up Apache}
        \textbf{attributes:}\\
        \small\texttt{default[:apache][:default\_modules] = ['php5']}\\
        \medskip
        \textbf{recipe:}\\
        \small\texttt{include\_recipe 'apache2'\\
            template "\#\{node[:apache][:dir]\}/sites-available/lamp-test.conf" do\\
            ~~source 'lamp-test.erb'\\
            ~~owner 'root'\\
            ~~group node[:apache][:root\_group]\\
            ~~mode '0644'\\
            ~~notifies :restart, 'service[apache2]'\\
            end\\
            \medskip
            apache\_site 'lamp-test' do\\
            ~~enable true\\
            end\\
        }
    \end{frame}
    \begin{frame}{Deploy Code}
        \texttt{include\_recipe 'git'\\
            \medskip
            deploy\_revision 'lamptest' do\\
            ~~revision 'master'\\
            ~~repository 'https://github.com/guywithnose/lamp-test.git'\\
            ~~deploy\_to '/var/www/lamp-test'\\
            ~~purge\_before\_symlink.clear()\\
            ~~create\_dirs\_before\_symlink.clear()\\
            ~~symlink\_before\_migrate.clear()\\
            ~~symlinks.clear()\\
            end
        }
    \end{frame}
\end{document}
